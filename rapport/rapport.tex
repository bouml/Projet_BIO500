\documentclass[9pt,twocolumn,twoside,]{pnas-new}

% Use the lineno option to display guide line numbers if required.
% Note that the use of elements such as single-column equations
% may affect the guide line number alignment.


\usepackage[T1]{fontenc}
\usepackage[utf8]{inputenc}

% tightlist command for lists without linebreak
\providecommand{\tightlist}{%
  \setlength{\itemsep}{0pt}\setlength{\parskip}{0pt}}


% Pandoc citation processing
\newlength{\cslhangindent}
\setlength{\cslhangindent}{1.5em}
\newlength{\csllabelwidth}
\setlength{\csllabelwidth}{3em}
\newlength{\cslentryspacingunit} % times entry-spacing
\setlength{\cslentryspacingunit}{\parskip}
% for Pandoc 2.8 to 2.10.1
\newenvironment{cslreferences}%
  {}%
  {\par}
% For Pandoc 2.11+
\newenvironment{CSLReferences}[2] % #1 hanging-ident, #2 entry spacing
 {% don't indent paragraphs
  \setlength{\parindent}{0pt}
  % turn on hanging indent if param 1 is 1
  \ifodd #1
  \let\oldpar\par
  \def\par{\hangindent=\cslhangindent\oldpar}
  \fi
  % set entry spacing
  \setlength{\parskip}{#2\cslentryspacingunit}
 }%
 {}
\usepackage{calc}
\newcommand{\CSLBlock}[1]{#1\hfill\break}
\newcommand{\CSLLeftMargin}[1]{\parbox[t]{\csllabelwidth}{#1}}
\newcommand{\CSLRightInline}[1]{\parbox[t]{\linewidth - \csllabelwidth}{#1}\break}
\newcommand{\CSLIndent}[1]{\hspace{\cslhangindent}#1}


\templatetype{pnasresearcharticle}  % Choose template

\title{Réseau de collaborations des étudiants du cours BIO500 de
l'Université de Sherbrooke donné à l'hiver 2022}

\author[a]{Alexandre Martineau}
\author[a]{Clara Prévosto}
\author[a]{Élisabeth Roy}
\author[a]{Laura Béland}
\author[a]{Laurence Boum}

  \affil[a]{Université de Sherbrooke, Départment de biologie, 2500
Boulevard de l'Université, Sherbrooke, Québec, J1K 2R1}


% Please give the surname of the lead author for the running footer
\leadauthor{}

% Please add here a significance statement to explain the relevance of your work
\significancestatement{}


\authorcontributions{}



\correspondingauthor{\textsuperscript{} }

% Keywords are not mandatory, but authors are strongly encouraged to provide them. If provided, please include two to five keywords, separated by the pipe symbol, e.g:
 \keywords{  Bacon
number |  Centralité |  Collaboration |  Liens |  Petits mondes  } 

\begin{abstract}
Dans ce rapport, le réseau de collaborations des étudiants du cours
BIO500 de l'Université de Sherbrooke, donné à l'hiver 2022, fut comparé
au réseau de type petit monde afin d'analyser si les caractéristique
d'un réseau de collaborations peut être considéré comme un réseau de
petit monde. L'utilisation de données recueillies sur 193 étudiants
concernant leurs liens entre chacun d'entre eux ont servis à construire
un réseau de collaborations. Les analyses effectuées sur ce réseau sont
l'analyse de la corrélation entre la centralité et le nombre de liens
des étudiants et le calcul du `Bacon number' d'une des étudiantes de
notre équipe. Ces analyses ont montré une corrélation non significative
entre la centralité et le nombre de liens et un `Bacon number' se
situant entre 2 et 4. Cependant, les réseaux de type petit monde ont une
corrélation entre leur centralité et leur nombre de liens et un `Bacon
number' de 6 et moins. En somme, ce réseau de collaborations n'a pas
toutes les caractéristiques étudiées dans ce rapport correspondant aux
caractéristiques des réseaux de type petit monde.
\end{abstract}

\dates{This manuscript was compiled on \today}
\doi{\url{www.pnas.org/cgi/doi/10.1073/pnas.XXXXXXXXXX}}

\begin{document}

% Optional adjustment to line up main text (after abstract) of first page with line numbers, when using both lineno and twocolumn options.
% You should only change this length when you've finalised the article contents.
\verticaladjustment{-2pt}



\maketitle
\thispagestyle{firststyle}
\ifthenelse{\boolean{shortarticle}}{\ifthenelse{\boolean{singlecolumn}}{\abscontentformatted}{\abscontent}}{}

% If your first paragraph (i.e. with the \dropcap) contains a list environment (quote, quotation, theorem, definition, enumerate, itemize...), the line after the list may have some extra indentation. If this is the case, add \parshape=0 to the end of the list environment.

\acknow{}

\hypertarget{introduction}{%
\section{Introduction}\label{introduction}}

Dans plusieurs études écologiques, les différents organismes d'un
environnement donné sont tous liés entre eux par différents liens qui
les unissent. Ces liens correspondent à leur réseau de connection et
permettent de voir l'impact qu'une espèce aurait sur le reste de la
population présente. Dans ce rapport, nous nous penchons sur un type de
réseau nommé le réseau de petit monde pour savoir si les
caractéristiques de ce réseau concordent avec celles du réseau de
collaborations que nous avons obtenu. Ce dernier a été produit à partir
d'une base de données des collaborations des étudiants du cours BIO500
donné à l'Université de Sherbrooke à l'hiver 2022. Afin de répondre à
notre objectif, nous cherchons plus précisement à savoir s'il existe une
corrélation entre le nombre de liens des étudiants entre eux et leur
centralité dans leur réseau de liens. Nous effectuons également le
calcul du Bacon number d'une des étudiante pour calculer la distance
entre cette étudiante et ses collègues du cours BIO500.

\hypertarget{muxe9thode}{%
\section{Méthode}\label{muxe9thode}}

Nous avons tout d'abord récolté les données des 193 étudiants reliés au
cours BIO500, soit des étudiants inscrits au cours et ayant collaboré
avec des étudiants inscrits au cours. Nous avons classé ces données en
trois tables, l'une concernant les étudiants avec des informations
telles que l'année de début de leurs études et le nom de leur programme
d'études, la deuxième concernant les cours avec des informations telles
que le sigle du cours et le nombre de crédits du cours et la dernière
concernant les collaborations entre les étudiants avec des informations
telles que le nom des deux étudiants et le sigle du cours dans lequel
ils ont collaboré. Nous avons utilisé le logiciel R version 4.1.3 pour
faire la lecture, le nettoyage, l'injection des données et l'analyse des
différentes données. À l'aide de ces données, nous avons généré trois
figures nous permettant de répondre à notre objectif, à savoir si notre
réseau se compare à un réseau de petit monde. Pour faciliter le lien
entre les données, l'analyse et les figures, nous avons créé un script
target. Nous avons également utilisé GitHub pour faciliter le partage et
le stockage des données.

\hypertarget{ruxe9sultats}{%
\section{Résultats}\label{ruxe9sultats}}

\begin{figure}
\centering
\includegraphics[width=0.5\textwidth,height=0.4\textheight]{../results/reseau.pdf}
\caption{Réseau de collaboration de la cohorte 2022 \label{fig:plot1}}
\end{figure}

La figure \ref{fig:plot1} montre le réseau de collaborations, soit la
position des étudiants en fonction du nombre de liens les uns par
rapport aux autres. On observe les points, représentant les étudiants,
reliés par leurs collaborations avec les autre points entre eux. Dans
cette figure, on retrouve des points plus gros plus au centre de la
figure représentant les étudants ayant le plus de collaborations et des
points plus petits vers l'extérieur de la figure représentant les
étudiants ayant moins de collaborations.

\begin{figure}
\centering
\includegraphics[width=0.5\textwidth,height=0.4\textheight]{../results/centralite.pdf}
\caption{Relation entre la centralité et le nombre de liens
\label{fig:plot2}}
\end{figure}

La figure \ref{fig:plot2} représente la corrélation entre la centralité
et le nombre de liens entre les étudiants. La corrélation effectuée pour
voir le lien entre ces deux facteurs n'est pas significative, car la
corrélation linéaire obtient une valeur de P de 0,2493
(P\textgreater0.05).

\begin{figure}
\centering
\includegraphics[width=0.5\textwidth,height=0.4\textheight]{../results/bacon.pdf}
\caption{Degrés de séparation d'Élisabeth Roy \label{fig:plot3}}
\end{figure}

La figure \ref{fig:plot3} représente le degré de séparation des
étudiants par rapport à l'étudiante Élisabeth Roy. On peut voir que la
majorité des étudiants ont une distance collaborative entre 2 et 4
degrés de séparation avec Élisabeth Roy.

\hypertarget{discussion}{%
\section{Discussion}\label{discussion}}

Le réseau que nous avons obtenu à la Figure \ref{fig:plot1} peut être
considéré comme un réseau de petit monde pour plusieurs raisons. Ce type
de réseau, dont l'exemple parfait est les réseaux sociaux, repose
principalement sur deux caractéristiques. D'abord, la distance moyenne
entre toutes les paires de nœuds d'un réseau de petit monde est faible
(1). Plus précisément, le degré de séparation moyens entre les individus
doit être aux alentours de six (1). Cette constatation, démontrée par le
sociologue Stanley Milgram, semble concorder avec notre réseau, dans
lequel on peut voir que les étudiants sont tous à quelques liens
seulement des autres étudiants. Pour un peu de perspective, Facebook a,
en 2011, rectifié le nombre de liens entre ses 2 milliards
d'utilisateurs : cette distance médiane est passé de six à seulement
quatre (2). Ensuite, pour qu'un réseau soit considéré comme un petit
monde, il faut qu'il contienne un niveau de regroupement (clustering)
local élevé, ce qui signifie que les nœuds sont généralement très
connectés à leurs voisins immédiats (1). On peut également observer
cette caractéristique sur notre réseau, principalement au centre, où
l'on trouve des points plus larges, signifiant qu'ils sont davantage
connectés à leurs voisins.

La Figure \ref{fig:plot2} indique le nombre de liens des étudiants par
rapport à leur centralité dans un réseau de type petit monde. Les
réseaux de petits monde sont des réseaux où les noeuds sont reliés les
un avec les autres avec, dans certains cas, des noeuds intermédiaires
pour permettre cette connection (3). Le degré de centralité est le
nombre de liens reliés à un noeud qui peut être traduit comme étant une
mesure d'influence directe d'un noeud (4). Ainsi, les noeuds
intermédiaires ayant plus de liens sont les plus influants. Cependant,
dans notre réseau, cette corrélation n'est pas significative. Il y a une
centralité plus forte chez les étudiants ayant une faible quantité de
liens. Ainsi, la corrélation entre la centralité et le nombre de liens
n'est pas la même que celle retrouvée dans un réseau de type petit
monde. Par contre, plusieurs relations existent par rapport à la mesure
de centralité. Faire l'étude de ses différents rapports pourrait aider à
comprendre la centralité obtenue dans notre figure.

La Figure \ref{fig:plot3} nous indique le nombre de nœuds séparant
Élisabeth Roy des autres étudiants. Ce nombre varie entre 1 et 4, la
majorité des étudiants ayant un degré de séparation entre 2 et 4 avec
Élisabeth Roy. Cette mesure se réfère directement au Bacon number ou au
Erdös number, basés tous deux sur le phénomène de petit monde. Ces deux
valeurs correspondent à la distance collaborative, d'un côté entre Kevin
Bacon et d'autres acteurs/actrices ayant collaboré dans des films, et de
l'autre entre le mathématicien Paul Erdos et d'autres auteurs/autrices
ayant participé à l'écriture d'articles académiques (5). Selon le
phénomène de petit monde, chaque personne aurait un degré de séparation
avec une autre personne de six maximum, c'est-à-dire que deux personnes
seraient liées par l'entremise de six personnes maximum. Cela concorde
donc avec le résultat que nous avons obtenu à la Figure \ref{fig:plot3},
soit qu'il y a 4 personnes au maximum séparant Élisabeth Roy des autres
étudiant(e)s. Toutefois, cette valeur ne prend en compte qu'une seule
étudiante. Il serait donc intéressant de faire la moyenne de ce même
calcul pour tous les étudiants afin de voir le degré de séparation moyen
entre les individus de notre réseau.

\hypertarget{conclusion}{%
\section{Conclusion}\label{conclusion}}

Le but de ce travail était d'établir si le réseau de collaborations des
étudiants du cours de BIO500 de l'Université de Sherbrooke à l'hiver
2022 concordait avec les caractéristiques d'un réseau de petit monde.
Pour ce faire, une figure représentant le réseau de collaborations des
étudiants entre eux fut produite. Cette figure démontrait une certaine
centralité au niveau des étudiants ayant un nombre de liens plus fort
par rapport à ceux ayant un nombre de liens plus faible. La corrélation
entre le nombre de liens par étudiant et leur centralité dans une figure
n'étant, cependant, pas significative. Par contre, le degré de
séparation, calculé sur un étudiant pris au hasard, correspond au nombre
de degré normalement retrouvé dans un réseau de petit monde. Il n'est
donc pas possible de dire que le réseau de collaboration des étudiants
du cours de BIO500 concorde avec toutes les caractéristiques d'un réseau
de petit monde. Cependant, des analyses plus poussées pourraient
permettre de mieux comparer ces types de réseaux.

\newpage

\hypertarget{bibliographie}{%
\section*{Bibliographie}\label{bibliographie}}
\addcontentsline{toc}{section}{Bibliographie}

\hypertarget{refs}{}
\begin{CSLReferences}{0}{0}
\leavevmode\vadjust pre{\hypertarget{ref-watts_collective_1998}{}}%
\CSLLeftMargin{1. }
\CSLRightInline{Watts DJ, Strogatz SH (1998)
\href{https://doi.org/10.1038/30918}{Collective dynamics of
{``small-world''} networks}. \emph{Nature} 393(6684):440--442.}

\leavevmode\vadjust pre{\hypertarget{ref-10.1145ux2f2380718.2380723}{}}%
\CSLLeftMargin{2. }
\CSLRightInline{Backstrom L, Boldi P, Rosa M, Ugander J, Vigna S (2012)
\href{https://doi.org/10.1145/2380718.2380723}{Four degrees of
separation}. \emph{Proceedings of the 4th Annual ACM Web Science
Conference}, WebSci '12. (Association for Computing Machinery, New York,
NY, USA), pp 33--42.}

\leavevmode\vadjust pre{\hypertarget{ref-uzzi2007small}{}}%
\CSLLeftMargin{3. }
\CSLRightInline{Uzzi B, Amaral LA, Reed-Tsochas F (2007) Small-world
networks and management science research: A review. \emph{European
Management Review} 4(2):77--91.}

\leavevmode\vadjust pre{\hypertarget{ref-borgatti2005centrality}{}}%
\CSLLeftMargin{4. }
\CSLRightInline{Borgatti SP (2005) Centrality and network flow.
\emph{Social networks} 27(1):55--71.}

\leavevmode\vadjust pre{\hypertarget{ref-collins1998s}{}}%
\CSLLeftMargin{5. }
\CSLRightInline{Collins JJ, Chow CC (1998) It's a small world.
\emph{Nature} 393(6684):409--410.}

\end{CSLReferences}



% Bibliography
% \bibliography{pnas-sample}

\end{document}
